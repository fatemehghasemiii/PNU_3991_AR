\documentclass [10pt,a4paper]{book}

\begin{document}
\begin{flushright}
\textbf{SURVEYS 151}
\end{flushright}
a survey participant request unless the subject line, the content of the message, and any
incentives both “hooks” and induces subjects to participate. Moreover, a recent survey
by Gilbert (2001) shows that as many as 85 percent of users, at least occasionally,
delete messages without reading them. A small number of users are also setting email
filters to eliminate postings from all but well-known senders. These options are mak-
ing it more difficult for the e-researcher to communicate with the targeted population.

\begin{flushleft}

 \textbf{Authenticity}
\end{flushleft}
Issues of authenticity plague all survey designs and may be exacerbated online. Unso-
phisticated designs make it difficult, or impossible, to determine if the participant
replying to the survey is the one who was sent the survey. Further, it may be difficult
to determine if subjects have replied multiple times to the e-survey.


\begin{flushleft}
\textbf{Security and Confidentiality}
\end{flushleft}
Issues of anonymity are also exacerbated online (at least in the minds of some poten-
tial respondents). The e-researcher is forced to create a trusting environment by
addressing directly issues of confidentiality, secure storage of results, and ethical
research behavior. These issues are discussed further in the ethics chapter (Chapter 5).

\begin{flushleft}
\textbf{Respondent Anger}
\end{flushleft}
Even the most well-crafted and inviting ¢-survey may be perceived by some respon-
dents as aggravating spam (unsolicited and unwanted email). The response of a recipi-
ent of a mail delivered paper survey is usually to throw it away or return it unanswered.
‘There is a small possibility that a disgruntled recipient of an e-survey may reply with
a virus or Trojan horse email bomb, or inappropriately forward, alter, or in other ways
misuse your e-survey. Until authentication and digital signatures become more wide-
spread, there is probably little that e-researchers can do to eliminate this problem,
other than to take standard procedures for protecting and checking their email and
Web sites for viruses or other malicious attacks
\begin{flushleft}
\textbf{Procrastination}
\end{flushleft}
The advantage of time shifting can also encourage procrastination. Some users have
noted the ease with which email can be glanced at and left unattended to at the bottom
of a growing list of emails, Provision of an attention-grabbing subject line is critical to
reduce this disadvantage.

\begin{flushleft}
\textbf{CRITICAL ISSUES IN e-SURVEY DESIGN AND ADMINISTRATION}
\end{flushleft}

In this section we look at several key design issues that every researcher must address
when using e-surveys. The task of the e-researcher is to minimize each of these errors
to the greatest degree possible within the constraints of the available time and budget.


\end{document}
