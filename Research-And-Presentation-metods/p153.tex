\documentclass [10pt,a4paper]{book}

\begin{document}
\begin{flushright}
 \textbf{SURVEYS 153}
\end{flushright}

\begin{flushleft}
\textbf{Nonresponse Error}
\end{flushleft}
Nonresponse error occurs when those who did not respond to the survey are in some
ways different from those who did respond and that difference is relevant to the
research study. An obvious example would be an e-survey to determine workload lev-
els of school principals. The principals with the heaviest workloads may be the ones
least likely to take the time to complete an e-survey and thus their critical information
will be lost, resulting in considerable nonresponse error.
\begin{flushleft}
\textbf{Response Bias}
\end{flushleft}
Response bias occurs when survey respondents deliberately or inadvertently falsify or
misrepresent their answers. Respondents may falsify answers to give socially acceptable
answers, to avoid potential embarrassment, or to conceal personal or confidential
information. Misrepresentations occur when respondents provide incorrect responses
to questions to which there is a correct answer.
\begin{flushleft}
\textbf{ACHIEVING A HIGH RESPONSE RATE}
\end{flushleft}
Although there is no absolute minimum for an acceptable response rate, the higher the
response rate, the more accurately the survey sample results will reflect the opinions of
the target population. Researchers use theories to help explain and predict a variety of
communication, interaction, and other human behaviors. For example, in the field of
social sciences, social exchange theory has been usefully adapted to provide guidelines
for the construction and administration of surveys (Dillman, 2000). Underlying this
theory is the premise that human behavior occurs and is channeled by the rewards that
result from these behaviors. If the behavior is to continue, the rewards to the individ-
ual must exceed the costs of engaging in the behavior. Further, since the rewards may
be long term or delayed in arriving, the participant must have trust (in the researcher)
that the benefits will outweigh the costs. In the following section, we describe the gen-
eral means by which these three important variables—rewards, risks, and trust—can be
used by the researcher to increase the response rates of e-surveys.
\begin{flushleft}
\textbf{Rewards}
\end{flushleft}
There are a variety of techniques by which the e-researcher can enhance the respon-
dents’ perception of reward for participating in the e-survey. Most obviously, the
e-researcher may wish to build in tangible incentives such as gift certificates,
promises of cash, discounts, or prizes. Reward is also engendered by the respondents’
perception that the survey is useful and worthwhile and that their participation in
the survey is important. Efforts should also be made to validate the position of
respondents by acknowledging their inclusion in the important group selected for
this study. Engaging participants immediately in the text of a cover letter and in the
first few questions is vitally important to this perception of reward. Engagement is

\end{document}
